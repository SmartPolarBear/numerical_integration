\documentclass[a4paper,twoside]{article}
\usepackage{blindtext}  
\usepackage{geometry}

% Chinese support
\usepackage[UTF8, scheme = plain]{ctex}

% Page margin layout
\geometry{left=2.3cm,right=2cm,top=2.5cm,bottom=2.0cm}


\usepackage{listings}
\usepackage{xcolor}
\usepackage{geometry}
\usepackage{amsmath}
\usepackage{float}
\usepackage{hyperref}

\usepackage{graphics}
\usepackage{graphicx}
\usepackage{subfigure}
\usepackage{epsfig}
\usepackage{float}

\usepackage{algorithm}
\usepackage[noend]{algpseudocode}

\usepackage{booktabs}
\usepackage{threeparttable}
\usepackage{longtable}
\usepackage{listings}
\usepackage{tikz}

% cite package, to clean up citations in the main text. Do not remove.
\usepackage{cite}

\usepackage{color,xcolor}

%% The amssymb package provides various useful mathematical symbols
\usepackage{amssymb}
%% The amsthm package provides extended theorem environments
\usepackage{amsthm}
\usepackage{amsfonts}
\usepackage{enumerate}
\usepackage{enumitem}
\usepackage{listings}

\usepackage{indentfirst}
\setlength{\parindent}{2em} % Make two letter space in the first paragraph
\usepackage{setspace}
\linespread{1.5} % Line spacing setting
\usepackage{siunitx}
\setlength{\parskip}{0.5em} % Paragraph spacing setting

% \usepackage[contents =22920202204622, scale = 10, color = black, angle = 50, opacity = .10]{background}

\newcommand*{\dif}{\mathop{}\!\mathrm{d}}

\renewcommand{\figurename}{图}
\renewcommand{\lstlistingname}{代码} 
\renewcommand{\tablename}{表格}
\renewcommand{\contentsname}{目录}
\floatname{algorithm}{算法}

\graphicspath{ {images/} }

%%%%%%%%%%%%%
\newcommand{\StudentNumber}{22920202204622}  % Fill your student number here
\newcommand{\StudentName}{熊恪峥}  % Replace your name here
\newcommand{\PaperTitle}{实验(三)\ \ 实现龙贝格积分}  % Change your paper title here
\newcommand{\PaperType}{计算方法(A)} % Replace the type of your report here
\newcommand{\Date}{2022年4月6日}
\newcommand{\College}{信息学院}
\newcommand{\CourseName}{计算方法(A)}
%%%%%%%%%%%%%

%% Page header and footer setting
\usepackage{fancyhdr}
\usepackage{lastpage}
\pagestyle{fancy}
\fancyhf{}
% This requires the document to be twoside
\fancyhead[LO]{\texttt{\StudentName }}
\fancyhead[LE]{\texttt{\StudentNumber}}
\fancyhead[C]{\texttt{\PaperTitle }}
\fancyhead[R]{\texttt{第{\thepage}页,共\pageref*{LastPage}页}}


\title{\PaperTitle}
\author{\StudentName}
\date{\Date}

\lstset{
	basicstyle          =   \sffamily,          % 基本代码风格
	keywordstyle        =   \bfseries,          % 关键字风格
	commentstyle        =   \rmfamily\itshape,  % 注释的风格,斜体
	stringstyle         =   \ttfamily,  % 字符串风格
	flexiblecolumns,                % 别问为什么,加上这个
	numbers             =   left,   % 行号的位置在左边
	showspaces          =   false,  % 是否显示空格,显示了有点乱,所以不现实了
	numberstyle         =   \zihao{-5}\ttfamily,    % 行号的样式,小五号,tt等宽字体
	showstringspaces    =   false,
	captionpos          =   t,      % 这段代码的名字所呈现的位置,t指的是top上面
	frame               =   lrtb,   % 显示边框
}

\lstdefinestyle{PythonStyle}{
	language        =   Python, % 语言选Python
	basicstyle      =   \zihao{-5}\ttfamily,
	numberstyle     =   \zihao{-5}\ttfamily,
	keywordstyle    =   \color{blue},
	keywordstyle    =   [2] \color{teal},
	stringstyle     =   \color{magenta},
	commentstyle    =   \color{red}\ttfamily,
	breaklines      =   true,   % 自动换行,建议不要写太长的行
	columns         =   fixed,  % 如果不加这一句,字间距就不固定,很丑,必须加
	basewidth       =   0.5em,
}

\algnewcommand\algorithmicinput{\textbf{Input:}}
\algnewcommand\algorithmicoutput{\textbf{Output:}}
\algnewcommand\Input{\item[\algorithmicinput]}%
\algnewcommand\Output{\item[\algorithmicoutput]}%

\usetikzlibrary{positioning, shapes.geometric}

\begin{document}
	
%%%%%%%%%%%%%%%%%%%%%%%%%%%%%%%%%%%%%%%%%%%%
\makeatletter % change default title style
\renewcommand*\maketitle{%
	\begin{center} 
		\bfseries  % title 
		{\LARGE \@title \par}  % LARGE typesetting
		\vskip 1em  %  margin 1em
		{\global\let\author\@empty}  % no author information
		{\global\let\date\@empty}  % no date
		\thispagestyle{empty}   %  empty page style
	\end{center}%
	\setcounter{footnote}{0}%
}
\makeatother
%%%%%%%%%%%%%%%%%%%%%%%%%%%%%%%%%%%%%%%%%%%%
	
	
\thispagestyle{empty}

\vspace*{1cm}

\begin{figure}[h]
	\centering
	\includegraphics[width=4.0cm]{logo.png}
\end{figure}

\vspace*{1cm}

\begin{center}
	\Huge{\textbf{\PaperType}}
	
	\Large{\PaperTitle}
\end{center}

\vspace*{1cm}

\begin{table}[h]
	\centering	
	\begin{Large}
		\renewcommand{\arraystretch}{1.5}
		\begin{tabular}{p{3cm} p{5cm}<{\centering}}
			姓\qquad 名 & \StudentName  \\
			\hline
			学\qquad号 & \StudentNumber \\
			\hline
			日\qquad期 & \Date  \\
			\hline
			学\qquad院 & \College  \\
			\hline
			课程名称 & \CourseName  \\
			\hline
		\end{tabular}
	\end{Large}
\end{table}

\newpage

\title{
	\Large{\textcolor{black}{\PaperTitle}}
}
	
	
\maketitle
	
\tableofcontents
 
\newpage
\setcounter{page}{1}

\begin{spacing}{1.2}

\section{实现}

\subsection{龙贝格积分法}

龙贝格积分法是通过外推法不断使用梯形公式来提高计算精度的数值积分法,它在区间上取等距离的点。它的过程如下:


\subsection{文件结构}

本程序实现了龙贝格积分以及用于基准测试和比较的蒙特卡洛积分法和拟蒙特卡洛(Quasi-Monte Carlo)积分法。
主要文件和作用如表格~\ref{tbl:struct}所示。

\begin{table}[htbp]
	\centering
	\renewcommand\arraystretch{1.5}
	\begin{tabular}{p{4cm}p{6cm}}
		\toprule
		文件 & 作用 \\
		\midrule
		integral/romberg.py & 龙贝格积分实现 \\
		\hline
		integral/qmc.py & 拟蒙特卡洛积分法实现 \\
		\hline
		integral/montecarlo.py & 蒙特卡洛积分法实现 \\
		\bottomrule
	\end{tabular}
	\label{tbl:struct}
	\caption{主要文件和作用}
\end{table}

本程序依赖如下两个第三方库:
\begin{itemize}
	\item \textbf{Numpy}: 用于加速向量运算
	\item \textbf{Scipy}: 用于生成低差异序列(Low-discrepancy sequence)
\end{itemize}

\subsection{接口实现}

本程序为龙贝格积分法实现以下接口,它支持通过给定任意可调用对象(Callable)表示的函数进行积分,具有良好的可扩展性。
\begin{lstlisting}[language=Python,numbers=left,style=PythonStyle,label={code:interface}]
def integrate(f: Callable[[np.float], np.float], a: np.float, b: np.float, epsilon: np.float = 0.001,
              n: int = 32) -> np.float:
\end{lstlisting}

该接口有如下参数:
\begin{itemize}
	\item $\mathbf{f}$: 被积函数的可调用对象。
	\item $\mathbf{a}, \mathbf{b}$: 表示积分区间$[a,b]$。
	\item $\mathbf{epsilon}$: 当两次迭代数值变化$\Delta T\le \varepsilon$时停止迭代,可以控制积分精度。
	\item  $\mathbf{n}$: 最大迭代次数。
\end{itemize}


\section{基准测试}

\section{结论}

导致龙贝格积分法的实现效率相对一般的原因可能是因为这个递推过程很难被向量化优化,
这导致了该实现没有有效利用现代硬件的优化。但是这造成的性能差异并不显著,可见
龙贝格积分法是一种非常高效的算法。

尽管在为了达到与蒙特卡洛积分法特别是使用了拉丁超立方采样的拟蒙特卡洛积分法可以比拟的精度,
需要较多的迭代次数从而影响了运行效率,但是考虑到拉丁超立方采样在多变量积分和图形渲染领域
的大量应用侧面证明的高效性和准确性,这也是正常的。


\clearpage

\appendix

\section{附录:蒙特卡洛积分法和拟蒙特卡洛积分}

使用蒙特卡洛积分是一种采用蒙特卡洛法估计积分的方法。设有积分\eqref{eqn:egint}

\begin{equation}
	\label{eqn:egint}
	I=\int_a^b f(x)d \dif x
\end{equation}

那么可以使用\eqref{eqn:mtest}估计这个积分。其中$pdf(X)$是概率密度函数。特别地
在实现中常取均匀分布$X\sim \mathcal{U}(a,b)$,则$pdf(x)=1/(b-a)$。

\begin{equation}
	\label{eqn:mtest}
	\bar{I_n}=F_n(X)=\frac{1}{n}\sum_{k=1}^n\frac{f(X_k)}{pdf(X_k)}
\end{equation}

\begin{proof}
	要证明这个估计是正确的,就需要证明该估计的数学期望和积分值相等。
\begin{align*}
	E[F_n]&=E\left[\frac{1}{n}\sum_{k=1}^n\frac{f(X_k)}{pdf(X_k)}\right]\\
	&=\frac{1}{n}\sum_{k=1}^n\int \frac{f(x)}{pdf(x)} \cdot pdf(x) \dif x\\
	&=\frac{1}{n}\sum_{k=1}^n\int f(x) \dif x\\
	&=\int f(x) \dif x
\end{align*}
\end{proof}
根据以上结论,可以得到算法~\ref{algo:mcint}

\begin{algorithm}
	\caption{蒙特卡洛法积分}
	\label{algo:mcint}
	\begin{algorithmic}[1]
		\Input{函数$F$,区间$[a,b]$,点数$n$}
		
		\Procedure{Integrate}{$F,a,b,n$}     
		\State $X=[x_1,x_2,x_3,\ldots], x_i \sim\mathcal{U}(a,b)$
		\State $Y=F(X)$
		\State \Return $(b-a)\frac{1}{n}\sum_{i=1}^nY_i$
		\EndProcedure
	\end{algorithmic}
\end{algorithm}	

作为一种随机算法,实际上在$n$较大时有较好的精度,它的收敛速度是$\mathcal{O}(n^{-0.5})$
如果采用\emph{低差异序列}替换纯随机数,积分收敛速度能够达到$\mathcal{O}(n^{-1})$,而且能够避免
随机采样中采样点的聚集。一种简单的一维低差异序列是\emph{Van Der Corput序列}。设有$b$进制数
$x=\sum_{k=0}^md_k\cdot b^k$,则对应的Van der Corput序列可由\eqref{eqn:vdc}计算。

\begin{equation}
	\label{eqn:vdc}
	g_b(x)=\sum_{k=0}^{m}d_k\cdot b^{-k-1}
\end{equation}
$g_b(x)$满足$g_b(x)\in [0,1]$,因此通过区间映射并替代算法~\ref{algo:mcint}中随机数可以得到
拟蒙特卡洛积分算法。此外,还可以使用Latin Hypercube Sampling(拉丁超立方采样)来替代随机数。

\clearpage

\section{附录:测试代码}
\label{sec:app_code}
\end{spacing}

\end{document}